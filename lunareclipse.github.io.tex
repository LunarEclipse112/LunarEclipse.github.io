%!TEX TS-program = xelatex
%!TEX encoding = UTF-8 Unicode
% Awesome CV LaTeX Template for CV/Resume
%
% This template has been downloaded from:
% https://github.com/posquit0/Awesome-CV
%
% Author:
% Claud D. Park <posquit0.bj@gmail.com>
% http://www.posquit0.com
%
%
% Adapted to be an Rmarkdown template by Mitchell O'Hara-Wild
% 23 November 2018
%
% Template license:
% CC BY-SA 4.0 (https://creativecommons.org/licenses/by-sa/4.0/)
%
%-------------------------------------------------------------------------------
% CONFIGURATIONS
%-------------------------------------------------------------------------------
% A4 paper size by default, use 'letterpaper' for US letter
\documentclass[11pt,a4paper,]{awesome-cv}

% Configure page margins with geometry
\usepackage{geometry}
\geometry{left=1.4cm, top=.8cm, right=1.4cm, bottom=1.8cm, footskip=.5cm}


% Specify the location of the included fonts
\fontdir[fonts/]

% Color for highlights
% Awesome Colors: awesome-emerald, awesome-skyblue, awesome-red, awesome-pink, awesome-orange
%                 awesome-nephritis, awesome-concrete, awesome-darknight

\colorlet{awesome}{awesome-red}

% Colors for text
% Uncomment if you would like to specify your own color
% \definecolor{darktext}{HTML}{414141}
% \definecolor{text}{HTML}{333333}
% \definecolor{graytext}{HTML}{5D5D5D}
% \definecolor{lighttext}{HTML}{999999}

% Set false if you don't want to highlight section with awesome color
\setbool{acvSectionColorHighlight}{true}

% If you would like to change the social information separator from a pipe (|) to something else
\renewcommand{\acvHeaderSocialSep}{\quad\textbar\quad}

\def\endfirstpage{\newpage}

%-------------------------------------------------------------------------------
%	PERSONAL INFORMATION
%	Comment any of the lines below if they are not required
%-------------------------------------------------------------------------------
% Available options: circle|rectangle,edge/noedge,left/right

\name{}{}



% \gitlab{gitlab-id}
% \stackoverflow{SO-id}{SO-name}
% \skype{skype-id}
% \reddit{reddit-id}


\usepackage{booktabs}

\providecommand{\tightlist}{%
	\setlength{\itemsep}{0pt}\setlength{\parskip}{0pt}}

%------------------------------------------------------------------------------



% Pandoc CSL macros
\newlength{\cslhangindent}
\setlength{\cslhangindent}{1.5em}
\newlength{\csllabelwidth}
\setlength{\csllabelwidth}{3em}
\newenvironment{CSLReferences}[3] % #1 hanging-ident, #2 entry spacing
 {% don't indent paragraphs
  \setlength{\parindent}{0pt}
  % turn on hanging indent if param 1 is 1
  \ifodd #1 \everypar{\setlength{\hangindent}{\cslhangindent}}\ignorespaces\fi
  % set entry spacing
  \ifnum #2 > 0
  \setlength{\parskip}{#2\baselineskip}
  \fi
 }%
 {}
\usepackage{calc}
\newcommand{\CSLBlock}[1]{#1\hfill\break}
\newcommand{\CSLLeftMargin}[1]{\parbox[t]{\csllabelwidth}{#1}}
\newcommand{\CSLRightInline}[1]{\parbox[t]{\linewidth - \csllabelwidth}{#1}}
\newcommand{\CSLIndent}[1]{\hspace{\cslhangindent}#1}

\begin{document}

% Print the header with above personal informations
% Give optional argument to change alignment(C: center, L: left, R: right)
\makecvheader

% Print the footer with 3 arguments(<left>, <center>, <right>)
% Leave any of these blank if they are not needed
% 2019-02-14 Chris Umphlett - add flexibility to the document name in footer, rather than have it be static Curriculum Vitae
\makecvfooter
  {2022-06-03}
    {~~~·~~~Curriculum Vitae}
  {\thepage}


%-------------------------------------------------------------------------------
%	CV/RESUME CONTENT
%	Each section is imported separately, open each file in turn to modify content
%------------------------------------------------------------------------------



\hypertarget{welcome}{%
\chapter{Welcome!}\label{welcome}}

Hi! And welcome, this is my portfolio. It is a collection of different
skills I posses and a way to show what I can do in a data analysis
context. Each page contains a different group of skills and stands alone
in its presentation. Also good to know is that this is of course, an
ongoing process.

\hypertarget{about-me}{%
\section{About Me}\label{about-me}}

I am a student of the university of applied sciences Utrecht. I have
great passion for biology and data analysis. A work enviorment where I
can implement both those interests is a dream job for me. With my
biology background and data analysis knowledge I can be a bridge between
lab analysts and data scientists. I am a transgender nonbinary
individual, This means I go by they/them (die/diens of hen/hun). Usually
this isn't something I place too much importance on, I just live my life
like anyone else but it might have some implications in the future
regarding out of the blue surgeries. I don't expect this to happen in
the following year but just in case I do feel like my future employer
should be informed about this.

\hypertarget{example-analysis-c.-elegans-experiment}{%
\chapter{\texorpdfstring{Example analysis \emph{C. Elegans}
experiment}{Example analysis C. Elegans experiment}}\label{example-analysis-c.-elegans-experiment}}

First we will look at a case analysis of a \emph{C. elegans} plate
experiment. The data for this experiment was kindly supplied by J.
Louter (INT/ILC). After exposure to various different compounds and a
certain incubation time \emph{C. elegans} nematode offspring where
counted to determine toxicity of the compounds. The data is supplied in
a excel file
\href{https://github.com/DataScienceILC/tlsc-dsfb26v-20_workflows/raw/main/data/CE.LIQ.FLOW.062_Tidydata.xlsx}{Data
Link}.

While looking at the data a number of things popped up. The compound
unit values aren't all the same. The values for the controls are listed
as being `pct', we will plot them in the same graph to see the
comparison for nematode offspring between the treated and control
groups. This wil allow us to make a guess at the efficacy of the
compounds in question. S-medium and Ethanol are the controls in this
study. Ethanol with a concentration of 1.5 pct is the positive control,
Medium-S is the negative control with a concentration of 0.0 pct and
lastly Ethanol with a concentration of 0.5 pct is the control vehicle A.
To better illustrate the position of the values in the graph a set
amount of jitter is added, 0.5 in the width and none in height, just to
pull apart the points for better visualization. All the y-axis values
remain thus unchanged and seeing as the concentration values are the
same within groups it should not disturb the data.

Looking at the graph we can estimate that there might be a correlation
between less offspring being counted on trated colonies of \emph{C.
Elegans} with higher concentrations of Naphtalene,
2,6-diisopropylnaphthalene, decane and Ethanol. To continue and test our
assumption the data will be normalized and plotted again using a min-max
scaling. This will allow us to better look at the the different
compounds compared to each other.

Based on the last graph we are able to say there might be a correlation
between the higher dosage of the different compounds (not controls,
except the positive control) and the reduced offspring in the different
\emph{C. elegans} populations counted.

\hypertarget{open-peer-review}{%
\chapter{Open Peer Review}\label{open-peer-review}}

Placeholder

\hypertarget{teaching-anxiety-stress-and-resilience-during-the-covid-19-pandemic}{%
\subsection{Teaching anxiety, stress and resilience during the COVID-19
pandemic}\label{teaching-anxiety-stress-and-resilience-during-the-covid-19-pandemic}}

\hypertarget{monitoring-trends-and-differences-in-covid-19-case-fatality-rates-using-decomposition-methods-contributions-of-age-structure-and-age-specific-fatality}{%
\subsection{Monitoring trends and differences in COVID-19 case-fatality
rates using decomposition methods: Contributions of age structure and
age-specific
fatality}\label{monitoring-trends-and-differences-in-covid-19-case-fatality-rates-using-decomposition-methods-contributions-of-age-structure-and-age-specific-fatality}}

\hypertarget{guerrilla-analytics}{%
\chapter{Guerrilla Analytics}\label{guerrilla-analytics}}

Placeholder

\hypertarget{looking-ahead}{%
\chapter{Looking ahead}\label{looking-ahead}}

To demonstrate my ability to learn new skills and continue to develop
myself I will try to learn as much as I can in four days about single
cell next generation sequencing.

I chose this skill because NGS is something that I find very interesting
and NGS from a single cell sounds awesome.

\hypertarget{plan}{%
\subsection{Plan}\label{plan}}

First a small overview of how I want to structure my 4 days to make sure
I best make use of the time. By the end of it I can hopefully run a
analysis, or part of one that isn't too complicated. To this end I will
need to read up about it and search for usable data used in research to
try and mimic what they have done. Hopefully in the future this will
allow me to beter develop the skills needed to run my own analysis on
Single Cell NGS data.

\begin{longtable}[]{@{}
  >{\raggedright\arraybackslash}p{(\columnwidth - 4\tabcolsep) * \real{0.0846}}
  >{\raggedright\arraybackslash}p{(\columnwidth - 4\tabcolsep) * \real{0.8308}}
  >{\raggedright\arraybackslash}p{(\columnwidth - 4\tabcolsep) * \real{0.0846}}@{}}
\toprule
\begin{minipage}[b]{\linewidth}\raggedright
Day Number
\end{minipage} & \begin{minipage}[b]{\linewidth}\raggedright
Activity to do
\end{minipage} & \begin{minipage}[b]{\linewidth}\raggedright
Completion
\end{minipage} \\
\midrule
\endhead
Day 1 & Reading about single cell NGS and looking for research articles
where this has been used for analysis & NOT DONE \\
Day 2 & Trying to get everything to work to mimic the analysis found in
the article & NOT DONE \\
Day 3 & Continue with the reproduction of the article analysis untill
done & NOT DONE \\
Day 4 & Write a report on what I have learned and things I need to learn
more about, a what's next secion let's say & NOT DONE \\
\bottomrule
\end{longtable}

\hypertarget{day-1}{%
\subsection{Day 1}\label{day-1}}

\hypertarget{projects}{%
\chapter{Projects}\label{projects}}

\hypertarget{noldus-project}{%
\section{Noldus project}\label{noldus-project}}

With Noldus Inc.~(Noldus \textbar{} {Advance} your behavioral research
(n.d.). In \emph{Noldus \textbar{} Advance your behavioral research}.
https://www.noldus.com.) and the University of Applied sciences Utrecht
\href{https://www.hu.nl/}{(website HU)} I worked in a team of 5 to
create a shiny app to help analyse data acquired from one of their
research products: The Erasmus Ladder. The Erasmus Ladder is a setup
that consists of a horizontal ladder to determine differences in
behavior of mice by recording steps made on a selection of sensors.
Variations in the amount of steps made and the type of steps that are
made by mice can be recorded very accurately which allows for
statistical analysis to determine differences between groups.

\hypertarget{the-application}{%
\subsection{The application}\label{the-application}}

Our application analyses data acquired from the Erasmus Ladder and plots
exploratory and publication ready graphs together with statistical
analysis for researchers to more easily interpret their results. One of
the challenges in creating the application was figuring out what kind of
data is interesting to researchers. By reading articles using the device
and speaking to researchers in the field we made a selection of
publication ready graph functions that read in data formatted by the
Erasmus ladder. we also made a selection of exploratory graphs meant to
help researches get an overview of their data by showing some basic
information like the amount of mice used, the different groups of mice,
etc. and some graphs that portray an overview of things like the
different types of steps mice made of different groups.

\hypertarget{example-graphs}{%
\subsection{Example graphs}\label{example-graphs}}

As examples of what data is interesting for research purposes we looked
at a selection of different articles. For us the most important criteria
for the articles was the use of the Erasmus ladder and we were mostly
interested in how they interpreted the data from it. The type of graphs
used and how they portrayed the data are very important for us to base
decisions on what graphs we want our application to produce based on
input data. One of the first graphs I found very interesting provides a
clear visual on the variances of steps made by different groups of mice
Vinueza Veloz, M. F., Zhou, K., Bosman, L. W. J., Potters, J.-W.,
Negrello, M., Seepers, R. M., Strydis, C., Koekkoek, S. K. E., \& De
Zeeuw, C. I. (2015). Cerebellar control of gait and interlimb
coordination. \emph{Brain Struct Funct}, \emph{220}(6), 3513--3536.
\url{https://doi.org/10.1007/s00429-014-0870-1}. In the graph there are
two different groups of mice, one with a Purkinje cell deficiency (Pcd)
and one control group that make runs crossing the Erasmus ladder. As
each step is recorded the type of steps is determined based on the
distance traveled. On page 3517 (Fig. 3) of the article we can observe a
difference in variation of the steps used which might be caused due to
the Purkinje cell deficiency. Versions of step type graphs plotted
against the session can also be found in research Sathyanesan, A., \&
Gallo, V. (2019). Cerebellar contribution to locomotor behavior: {A}
neurodevelopmental perspective. \emph{Neurobiology of Learning and
Memory}, \emph{165}, 106861.
\url{https://doi.org/10.1016/j.nlm.2018.04.016} Sathyanesan, A., Kundu,
S., Abbah, J., \& Gallo, V. (2018). Neonatal brain injury causes
cerebellar learning deficits and {Purkinje} cell dysfunction. \emph{Nat
Commun}, \emph{9}(1), 3235.
\url{https://doi.org/10.1038/s41467-018-05656-w}. The graphs depict a
decrease in variation of steps that are used by the control groups in
time. This might be caused by the mice learning to cross the bridge more
efficiently. A different type of graph which looked at the reaction of
mice to the different ques to guide the mice on and off the device. This
research Vinueza Veloz, M. F., Buijsen, R. A. M., Willemsen, R., Cupido,
A., Bosman, L. W. J., Koekkoek, S. K. E., Potters, J. W., Oostra, B. A.,
\& De Zeeuw, C. I. (2012). The effect of an {mGluR5} inhibitor on
procedural memory and avoidance discrimination impairments in
{\emph{Fmr1}} {KO} mice. \emph{Genes, Brain and Behavior}, \emph{11}(3),
325--331. \url{https://doi.org/10.1111/j.1601-183X.2011.00763.x} gave us
a fresh look at the data and helped guide our decisions to what data
might be seen as relevant.

We also looked at other types of cerebellum research Schonewille, M.,
Gao, Z., Boele, H.-J., Vinueza Veloz, M. F., Amerika, W. E., Šimek, A.
A. M., De Jeu, M. T., Steinberg, J. P., Takamiya, K., Hoebeek, F. E.,
Linden, D. J., Huganir, R. L., \& De Zeeuw, C. I. (2011). Reevaluating
the {Role} of {LTD} in {Cerebellar Motor Learning}. \emph{Neuron},
\emph{70}(1), 43--50. \url{https://doi.org/10.1016/j.neuron.2011.02.044}
and how they selected data to better answer research questions. Although
mostly not very applicable in our project it did give some insight in
how slight differences can be.

\hypertarget{resume}{%
\chapter{Resume}\label{resume}}

Most likely if you are here you already received my C.V. but in case you
haven't here it is.

\hypertarget{references}{%
\chapter{References}\label{references}}

\hypertarget{pages}{%
\section{404 pages}\label{pages}}

By default, users will be directed to a 404 page if they try to access a
webpage that cannot be found.

\hypertarget{references-1}{%
\section{References}\label{references-1}}

\hypertarget{refs}{}
\begin{CSLReferences}{0}{0}
\leavevmode\vadjust pre{\hypertarget{ref-noauthor_noldus_nodate}{}}%
\CSLLeftMargin{1. }
\CSLRightInline{Noldus \textbar{} {Advance} your behavioral research.
(n.d.). In \emph{Noldus \textbar{} Advance your behavioral research}.
https://www.noldus.com.}

\leavevmode\vadjust pre{\hypertarget{ref-vinueza_veloz_cerebellar_2015}{}}%
\CSLLeftMargin{2. }
\CSLRightInline{Vinueza Veloz, M. F., Zhou, K., Bosman, L. W. J.,
Potters, J.-W., Negrello, M., Seepers, R. M., Strydis, C., Koekkoek, S.
K. E., \& De Zeeuw, C. I. (2015). Cerebellar control of gait and
interlimb coordination. \emph{Brain Struct Funct}, \emph{220}(6),
3513--3536. \url{https://doi.org/10.1007/s00429-014-0870-1}}

\leavevmode\vadjust pre{\hypertarget{ref-sathyanesan_cerebellar_2019}{}}%
\CSLLeftMargin{3. }
\CSLRightInline{Sathyanesan, A., \& Gallo, V. (2019). Cerebellar
contribution to locomotor behavior: {A} neurodevelopmental perspective.
\emph{Neurobiology of Learning and Memory}, \emph{165}, 106861.
\url{https://doi.org/10.1016/j.nlm.2018.04.016}}

\leavevmode\vadjust pre{\hypertarget{ref-sathyanesan_neonatal_2018}{}}%
\CSLLeftMargin{4. }
\CSLRightInline{Sathyanesan, A., Kundu, S., Abbah, J., \& Gallo, V.
(2018). Neonatal brain injury causes cerebellar learning deficits and
{Purkinje} cell dysfunction. \emph{Nat Commun}, \emph{9}(1), 3235.
\url{https://doi.org/10.1038/s41467-018-05656-w}}

\leavevmode\vadjust pre{\hypertarget{ref-vinueza_veloz_effect_2012}{}}%
\CSLLeftMargin{5. }
\CSLRightInline{Vinueza Veloz, M. F., Buijsen, R. A. M., Willemsen, R.,
Cupido, A., Bosman, L. W. J., Koekkoek, S. K. E., Potters, J. W.,
Oostra, B. A., \& De Zeeuw, C. I. (2012). The effect of an {mGluR5}
inhibitor on procedural memory and avoidance discrimination impairments
in {\emph{Fmr1}} {KO} mice. \emph{Genes, Brain and Behavior},
\emph{11}(3), 325--331.
\url{https://doi.org/10.1111/j.1601-183X.2011.00763.x}}

\leavevmode\vadjust pre{\hypertarget{ref-schonewille_reevaluating_2011}{}}%
\CSLLeftMargin{6. }
\CSLRightInline{Schonewille, M., Gao, Z., Boele, H.-J., Vinueza Veloz,
M. F., Amerika, W. E., Šimek, A. A. M., De Jeu, M. T., Steinberg, J. P.,
Takamiya, K., Hoebeek, F. E., Linden, D. J., Huganir, R. L., \& De
Zeeuw, C. I. (2011). Reevaluating the {Role} of {LTD} in {Cerebellar
Motor Learning}. \emph{Neuron}, \emph{70}(1), 43--50.
\url{https://doi.org/10.1016/j.neuron.2011.02.044}}

\end{CSLReferences}



\end{document}
